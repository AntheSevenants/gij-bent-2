% Options for packages loaded elsewhere
\PassOptionsToPackage{unicode}{hyperref}
\PassOptionsToPackage{hyphens}{url}
\PassOptionsToPackage{dvipsnames,svgnames,x11names}{xcolor}
%
\documentclass[
  letterpaper,
  DIV=11,
  numbers=noendperiod,
  oneside]{scrartcl}

\usepackage{amsmath,amssymb}
\usepackage{iftex}
\ifPDFTeX
  \usepackage[T1]{fontenc}
  \usepackage[utf8]{inputenc}
  \usepackage{textcomp} % provide euro and other symbols
\else % if luatex or xetex
  \usepackage{unicode-math}
  \defaultfontfeatures{Scale=MatchLowercase}
  \defaultfontfeatures[\rmfamily]{Ligatures=TeX,Scale=1}
\fi
\usepackage{lmodern}
\ifPDFTeX\else  
    % xetex/luatex font selection
\fi
% Use upquote if available, for straight quotes in verbatim environments
\IfFileExists{upquote.sty}{\usepackage{upquote}}{}
\IfFileExists{microtype.sty}{% use microtype if available
  \usepackage[]{microtype}
  \UseMicrotypeSet[protrusion]{basicmath} % disable protrusion for tt fonts
}{}
\makeatletter
\@ifundefined{KOMAClassName}{% if non-KOMA class
  \IfFileExists{parskip.sty}{%
    \usepackage{parskip}
  }{% else
    \setlength{\parindent}{0pt}
    \setlength{\parskip}{6pt plus 2pt minus 1pt}}
}{% if KOMA class
  \KOMAoptions{parskip=half}}
\makeatother
\usepackage{xcolor}
\usepackage[left=1in,marginparwidth=2.0666666666667in,textwidth=4.1333333333333in,marginparsep=0.3in]{geometry}
\setlength{\emergencystretch}{3em} % prevent overfull lines
\setcounter{secnumdepth}{5}
% Make \paragraph and \subparagraph free-standing
\makeatletter
\ifx\paragraph\undefined\else
  \let\oldparagraph\paragraph
  \renewcommand{\paragraph}{
    \@ifstar
      \xxxParagraphStar
      \xxxParagraphNoStar
  }
  \newcommand{\xxxParagraphStar}[1]{\oldparagraph*{#1}\mbox{}}
  \newcommand{\xxxParagraphNoStar}[1]{\oldparagraph{#1}\mbox{}}
\fi
\ifx\subparagraph\undefined\else
  \let\oldsubparagraph\subparagraph
  \renewcommand{\subparagraph}{
    \@ifstar
      \xxxSubParagraphStar
      \xxxSubParagraphNoStar
  }
  \newcommand{\xxxSubParagraphStar}[1]{\oldsubparagraph*{#1}\mbox{}}
  \newcommand{\xxxSubParagraphNoStar}[1]{\oldsubparagraph{#1}\mbox{}}
\fi
\makeatother


\providecommand{\tightlist}{%
  \setlength{\itemsep}{0pt}\setlength{\parskip}{0pt}}\usepackage{longtable,booktabs,array}
\usepackage{calc} % for calculating minipage widths
% Correct order of tables after \paragraph or \subparagraph
\usepackage{etoolbox}
\makeatletter
\patchcmd\longtable{\par}{\if@noskipsec\mbox{}\fi\par}{}{}
\makeatother
% Allow footnotes in longtable head/foot
\IfFileExists{footnotehyper.sty}{\usepackage{footnotehyper}}{\usepackage{footnote}}
\makesavenoteenv{longtable}
\usepackage{graphicx}
\makeatletter
\def\maxwidth{\ifdim\Gin@nat@width>\linewidth\linewidth\else\Gin@nat@width\fi}
\def\maxheight{\ifdim\Gin@nat@height>\textheight\textheight\else\Gin@nat@height\fi}
\makeatother
% Scale images if necessary, so that they will not overflow the page
% margins by default, and it is still possible to overwrite the defaults
% using explicit options in \includegraphics[width, height, ...]{}
\setkeys{Gin}{width=\maxwidth,height=\maxheight,keepaspectratio}
% Set default figure placement to htbp
\makeatletter
\def\fps@figure{htbp}
\makeatother
% definitions for citeproc citations
\NewDocumentCommand\citeproctext{}{}
\NewDocumentCommand\citeproc{mm}{%
  \begingroup\def\citeproctext{#2}\cite{#1}\endgroup}
\makeatletter
 % allow citations to break across lines
 \let\@cite@ofmt\@firstofone
 % avoid brackets around text for \cite:
 \def\@biblabel#1{}
 \def\@cite#1#2{{#1\if@tempswa , #2\fi}}
\makeatother
\newlength{\cslhangindent}
\setlength{\cslhangindent}{1.5em}
\newlength{\csllabelwidth}
\setlength{\csllabelwidth}{3em}
\newenvironment{CSLReferences}[2] % #1 hanging-indent, #2 entry-spacing
 {\begin{list}{}{%
  \setlength{\itemindent}{0pt}
  \setlength{\leftmargin}{0pt}
  \setlength{\parsep}{0pt}
  % turn on hanging indent if param 1 is 1
  \ifodd #1
   \setlength{\leftmargin}{\cslhangindent}
   \setlength{\itemindent}{-1\cslhangindent}
  \fi
  % set entry spacing
  \setlength{\itemsep}{#2\baselineskip}}}
 {\end{list}}
\usepackage{calc}
\newcommand{\CSLBlock}[1]{\hfill\break\parbox[t]{\linewidth}{\strut\ignorespaces#1\strut}}
\newcommand{\CSLLeftMargin}[1]{\parbox[t]{\csllabelwidth}{\strut#1\strut}}
\newcommand{\CSLRightInline}[1]{\parbox[t]{\linewidth - \csllabelwidth}{\strut#1\strut}}
\newcommand{\CSLIndent}[1]{\hspace{\cslhangindent}#1}

\KOMAoption{captions}{tablesignature}
\makeatletter
\@ifpackageloaded{caption}{}{\usepackage{caption}}
\AtBeginDocument{%
\ifdefined\contentsname
  \renewcommand*\contentsname{Table of contents}
\else
  \newcommand\contentsname{Table of contents}
\fi
\ifdefined\listfigurename
  \renewcommand*\listfigurename{List of Figures}
\else
  \newcommand\listfigurename{List of Figures}
\fi
\ifdefined\listtablename
  \renewcommand*\listtablename{List of Tables}
\else
  \newcommand\listtablename{List of Tables}
\fi
\ifdefined\figurename
  \renewcommand*\figurename{Figure}
\else
  \newcommand\figurename{Figure}
\fi
\ifdefined\tablename
  \renewcommand*\tablename{Table}
\else
  \newcommand\tablename{Table}
\fi
}
\@ifpackageloaded{float}{}{\usepackage{float}}
\floatstyle{ruled}
\@ifundefined{c@chapter}{\newfloat{codelisting}{h}{lop}}{\newfloat{codelisting}{h}{lop}[chapter]}
\floatname{codelisting}{Listing}
\newcommand*\listoflistings{\listof{codelisting}{List of Listings}}
\makeatother
\makeatletter
\makeatother
\makeatletter
\@ifpackageloaded{caption}{}{\usepackage{caption}}
\@ifpackageloaded{subcaption}{}{\usepackage{subcaption}}
\makeatother
\makeatletter
\@ifpackageloaded{sidenotes}{}{\usepackage{sidenotes}}
\@ifpackageloaded{marginnote}{}{\usepackage{marginnote}}
\makeatother

\ifLuaTeX
  \usepackage{selnolig}  % disable illegal ligatures
\fi
\usepackage{bookmark}

\IfFileExists{xurl.sty}{\usepackage{xurl}}{} % add URL line breaks if available
\urlstyle{same} % disable monospaced font for URLs
\hypersetup{
  pdftitle={gij bent 2},
  pdfauthor={Anthe Sevenants; Julie Nijs},
  colorlinks=true,
  linkcolor={blue},
  filecolor={Maroon},
  citecolor={Blue},
  urlcolor={Blue},
  pdfcreator={LaTeX via pandoc}}


\title{gij bent 2}
\author{Anthe Sevenants \and Julie Nijs}
\date{}

\begin{document}
\maketitle
\begin{abstract}
TODO
\end{abstract}

\renewcommand*\contentsname{Table of contents}
{
\hypersetup{linkcolor=}
\setcounter{tocdepth}{4}
\tableofcontents
}

\section{Introduction}\label{sec-introduction}

One of the key ideas from usage-based linguistics is that language is
not a system that exists just ``by itself''. Rather, it is widely
accepted within this branch that language arises out of the many
interactions that language users have with each other (Keller 1994). As
such, language becomes a ``complex-adaptive system'' (Ellis 2012;
Kretzschmar 2015), a system without a single authoritative control,
constructed through the simple yet numerous interactions of its
components, language users. While each speaker of a language (be it
Dutch, English or Maori) has their own personalised representation of
the rules and constraints that they think make up that language, these
representations have enough in common in order to warrant talking about
a shared ``code'', or more broadly, a language.

When it comes to research of said shared code, things become more
complicated. It is impossible for researchers to know the complete code
of \emph{all} language users in a language system. This situation forces
models of language into necessary generalisations. Generalisation is a
valid strategy, as most linguistic variation, when tallied at large,
will lead to a Zipfian (Zipf 1965) A-curve (as suggested by Kretzschmar
2015). The make-up of this curve, with the most popular forms capturing
most of the language use anyway, makes it so the ``general'' behaviour
of language variation that is typically captured in corpora also
represents a fair share of what would be the individual code of most
language users. Of course, there are factors influencing what language
forms are used,\sidenote{\footnotesize For example, one of the language varieties or
  ``lects'', defined as ``dialects, regiolects, national varieties,
  registers, styles, idiolects, whatever'' (Geeraerts 2005, 168)} but
broadly speaking, this generalisation holds. We see this, for example,
in probabilistic grammar, the branch of variation studies that
investigates how variation reflects competing constraints on linguistic
choices (Grafmiller et al. 2018). Still, on a theoretical level, it is
regrettable that models of variation do not start from the lowest common
denominator of this variation: the language user.

Contrastively, the advent of social media offers a unique opportunity
for variationist research. Social media platforms offer an insight into
the language systems of a large number of people. Whereas previously it
was possible to gauge the linguistic profile of only a small group of
people (e.g. Labov et al. 1968; Milroy and Milroy 1978), social media
offer linguistic variation at an unprecedented scale.\sidenote{\footnotesize More
  recently, the release of the C-Clamp corpus {[}todo{]} offers a
  diachronic insight into the language use of Dutch-speaking writers
  from XXX to XXX. The historical nature of this corpus, however, limits
  its use for research into contemporary language variation phenomena.}
While one should not consider spoken and written social media as
equivalent, social media language has nonetheless proven a useful proxy
for the former (Vandekerckhove and Cuvelier 2007, 255). In addition,
social media data of course do not provide access to \emph{all} of
language for \emph{all} users of a language system; not all utterances
are made on social media, nor are all language users active on social
media platforms. Nevertheless, social media platforms bring us closer to
the idea of studying language \emph{inside out}, starting from
individual language users' variation, even if it is only for a single
alternation among a subset of language users. This brings us to the
first research question of this study:

\begin{enumerate}
\def\labelenumi{\arabic{enumi}.}
\tightlist
\item
  What affordances does a focus on individual variation bring us in the
  context of alternation research, especially in contrast to
  traditional, aggregated alternation research?
\end{enumerate}

{[} @ Julie: misschien hier een stuk over het reeds bestaande onderzoek
naar individuele variatie. hoeft niet veel te zijn maar nu staat er
niets en dat is misschien niet top {]}

Another aspect of of the complex-adaptive system of language are the
connections between people. Humans are embedded in larger social
networks, which are formed according to a so-called ``power law''. This
means the network has a small number of people with a large number of
connections, supplemented by many people who have relatively few
connections (Barabasi 1999).

Linguistically, the effects of networks has been studied before {[}stuk
Julie{]}.

\begin{itemize}
\tightlist
\item
  leaders / loaners
\item
  gedrag van centrale figuren
\item
  alle andere dingen die interessant zouden kunenn zijn
\end{itemize}

Except for small-scale studies (e.g. Labov et al. 1968; Milroy and
Milroy 1978), it has been impossible to get an elaborate insight into
the social relationships of people on a large scale, let alone how these
relationships relate to language. Again, social media offer an
opportunity here. While it is not practical to get a full network of
\emph{all} social media users, we \emph{can} get information on how
``connected'' a particular user is. Through metrics like ``followers''
and ``following'', we know how many connections social media users have.
This allows us to get an idea of the centrality of a user in the larger
social media network, even if we do not know the full network itself.
This brings us to the second research question in this study:

\begin{enumerate}
\def\labelenumi{\arabic{enumi}.}
\setcounter{enumi}{1}
\tightlist
\item
  How can we leverage the network implicit in social media data to learn
  more about the spread of linguistic innovations?
\end{enumerate}

To answer the two research questions stated above, we follow up on a
study by Sevenants (Sevenants 2024a) about \emph{gij bent}, a linguistic
innovation concerning the second person of `to be' in Colloquial Belgian
Dutch (CBD). To compute the individual variation of different language
users, we will use elastic net regression, a technique suited to
determining individual variation. First, we will introduce the \emph{gij
bent} CBD linguistic innovation. Then, we will introduce the dataset and
describe how we modelled the linguistic profiles of the individual users
with regard to this innovation using elastic net, and why we need this
specific technique to answer the questions we have in this study. Next,
we will show the results of our analyses, and discuss what consequences
they have for usage-based alternation research in general.

\section{\texorpdfstring{The case of \emph{gij
bent}}{The case of gij bent}}\label{the-case-of-gij-bent}

This study focusses on \emph{gij bent}, an innovation within Colloquial
Belgian Dutch (CBD). Before we explain what this innovation is about, we
will first explain the general language situation of Dutch in Belgium.

\subsection{Dutch in Belgium}\label{dutch-in-belgium}

Dutch as it is spoken in Belgium has evolved strongly since the second
half of the 20\textsuperscript{th} century. In the largest part of the
20\textsuperscript{th} century, the language situation of Dutch in
Flanders, the Dutch-speaking part of Belgium, could best be described as
a \emph{diglossia} (Auer 2005). This means that the Dutch was
characterised by a dichotomy between standard language on the one hand,
and dialects on the other hand. These two were the language modes
available to speakers of Dutch. A large-scale language policy campaign
at the timed attempted to teach those speakers who mastered only their
local dialect the standard language. While these efforts were well
intended, the imposition of a ``foreign'' standard language -- the
standard was imported from the Netherlands -- were met with limited
enthusiasm by language users in Flanders. However, this language
campaign was not without effect, albeit not the intended one.

At the start of the 21\textsuperscript{st} century, it had become clear
that Flanders was evolving into a new type of language situation:
\emph{diaglossia} (Auer 2005). In between the dialects and standard
language, new language types had appeared, among which so-called
``regiolects'' (above the dialects) and ``regional standards'' (below
standard language). More generally, however, a new term was coined:
\emph{tussentaal}, which roughly translates to ``interlanguage''. In
this article, we will use the English term as defined by Vandekerckhove
(2005): ``Colloquial Belgian Dutch'' or CBD.

The new variant CBD is characterised by being hard to define. In
general, it is best described as a ``standard-adjacent'', colloquial
form of Belgian Standard Dutch, with both general Flemish elements and
region-specific features added to it on all levels (phonology,
morphology, syntax etc.). The presence of region-specific features means
that CBD is not a monolithic language variety, but rather sounds
different across all places in Flanders. This inherent variability is
intrinsic to its linguistic identity, but frustrating to linguists, who
have been trying to come up with ``defining features'' for decades (i.e.
Geeraerts 2000; Taeldeman 2008).

\subsection{\texorpdfstring{\emph{gij bent}}{gij bent}}\label{gij-bent}

Traditionally in Flanders, if one wishes to say ``you are'' in an
informal way, they would say \emph{gij zijt}. \emph{Gij} marks the
informal second person pronoun that is typical to most Flemish dialects.
\emph{Zijt} is a historic form of \emph{zijn} `to be' that is now
exclusively used with this pronoun. If one wanted to say ``you are'' in
a formal way (i.e.~in standard language), they would say \emph{jij
bent}. \emph{Jij} marks the non-polite, yet formal second person pronoun
in Flanders. \emph{Jij} is more formal than \emph{gij} because the
\emph{jij} pronoun is actually a Netherlandic form that, because it is
foreign, has a different status than \emph{gij}. \emph{Bent} is the
conjugated form that goes along with \emph{jij}. It is also imported
from the Netherlands.

It is clear, then, that these two forms are distinctly different from
each other. However, in CBD, a blend of these two forms has appeared:
\emph{gij bent}. It combines the \emph{gij} pronoun from \emph{gij zijt}
with the \emph{bent} conjugation known from standard language. \emph{Gij
bent} is a quite literal example of the diaglossia situation, as it
combines \emph{gij} and \emph{bent} from polar opposites of the language
spectrum and meets somewhere in the middle.

Research by Sevenants (2024a) has shown that \emph{gij bent} most likely
originated in dialects from the northern Antwerp province in Flanders,
and then spread to other parts of the Brabantian dialect
area.\sidenote{\footnotesize Brabants is the dialect area coinciding, mostly, with the
  Flemish provinces ``Vlaams-Brabant'' and ``Antwerp''. It is said to be
  the driving force behind CBD, but this claim is disputed (Ghyselen,
  Anne-Sophie 2016).} The reason for the spread from this area is argued
to lie in the function of \emph{gij bent}. The \emph{jij bent} from
standard language might feel too formal, while the \emph{gij zijt} from
dialects might feel too blunt. \emph{Gij bent} strikes a balance between
the two: it has colloquial roots through Flemish \emph{gij}, but
benefits from the prestige associated with standard language (Impe and
Speelman 2007) through standard \emph{bent}. In addition, both regional,
formality and gender effects were found:

\begin{enumerate}
\def\labelenumi{\arabic{enumi}.}
\tightlist
\item
  the use of \emph{gij bent} is limited to the Brabant dialect region
  only
\item
  the use of \emph{gij bent} is less likely in more informal contexts
\item
  the use of \emph{gij bent} is more likely for women
\end{enumerate}

While these conclusions are drawn from language use originating from
social media, the analysis itself is very traditional. As in a standard
corpus linguistic study, it generalises over all language at once, as
well as all language users. This leads to blanket statements as the ones
above. As argued before, there is nothing inherently wrong with this
approach, but given the opportunities in social media data, a more
bottom-up approach would be interesting. More specifically, such an
approach would allow us to detect whether certain language users behave
differently from what we would expect given their location or gender.
These deviations from general expectations are now invisible in the
generalising models. In addition, no account was taken of the network
structure hidden in the data, which might reveal more about the forces
and the direction this innovation will take in the future.

\section{Data and methodology}\label{data-and-methodology}

\subsection{Dataset}\label{dataset}

As this study is a follow-up on the \emph{gij bent} study by Sevenants
(2024a), we will re-use the dataset used in that study. The dataset
contains 29,429 social media posts from Belgium on the Twitter platform
(now X). These tweets either contain the traditional \emph{gij zijt} or
the innovative \emph{gij bent} construction. The tweets are tagged for
the Flemish dialect region to which each tweet author belongs. In
addition, gender information was estimated for each tweet author using
\emph{gender-from-name-r} (Sevenants 2024b). Authors for whom gender
could not be established were removed, to retain parity with Sevenants
(2024a). The dataset also contains a formality distinction for each
tweet (``more formal'', ``more informal''), based on whether a tweet
interacts with other people. If it does, the idea is that the tweet is
conversation-like and thus more informal. Each tweet also has a unique
(anonymised) identifier, so tweets can be grouped by author. This will
prove especially useful for this study on individual variation, since we
need several tweets in order to gauge the language use of a specific
speaker. For each author, we also know many accounts they follow, and
how many accounts they are followed by. This is a metric that is useful
for our questions about network influences. For our analysis of
individual variation, we retained only those authors who have at least
10 tweets in our dataset. We did this in order to guarantee a stable
estimation of each speaker's language use. After all filtering
operations, we were left with 16,968 tweets by 635 unique authors.

\subsection{Elastic net regression}\label{elastic-net-regression}

To estimate the individual variation among our tweet authors, we would
ideally have a quantitative measure which indicates the deviation of
each user from the expected linguistic behaviour. Authors who do not
deviate would be assigned a (near-)zero value, while those who \emph{do}
deviate would receive either a positive or negative value, depending on
whether they exceed the expectations for their profiles, or actually go
against it. In this way, our measure of individual variation truly
expresses the peculiarities of each individual speaker.

In order to operationalise this computation of individual variation, we
made use of elastic net regression (Friedman, Tibshirani, and Hastie
2010). Elastic net regression is an extension of more ``traditional''
regression techniques (discussed in Gries 2015). Its main innovation is
that it uses an extra ``penalty'' term, which prevents model
coefficients, the values associated with model predictors, from growing
too large. One could think of this as making the model \emph{worse}, but
on purpose. In practice, this model penalty is applied for two reasons.
First, it attempts to prevent overfitting. Too few data points can skew
the estimation of a coefficient, so the penalty is a means to attenuate
predictions in such cases. Second, the model penalty can function as a
way to do variable selection. With elastic net, a coefficient can be
``punished'' to the extent that it nears or equals to zero. With this,
the predictor of that coefficient is disabled outright. We can use this
behaviour to enter many predictors in the same analysis at once, and let
the model decide which ones are actually interesting.

In addition, elastic net leverages ``k-fold cross validation''. This
technique divides the data into several parts and iteratively uses
different parts for training and testing the regression model. This
practice also contributes towards general robustness and the prevention
of overfitting. K-fold cross validation allows us to have hundreds of
predictors in a single analysis, which is a useful property when we want
to track a multitude of language users at the same time. Such a thing
would be impossible with a regular regression analysis on the basis of
mathematical intractability (i.e.~with mixed models). For additional
details on elastic net regression for linguistics, see Sevenants, Van de
Velde, and Speelman (n.d.).

In general, then, elastic net regression lends itself nicely to research
into individual variation. With elastic net, we modelled the presence of
either \emph{zijt} (conservative) or \emph{bent} (innovative) as a
logistic response variable. In our model, we estimated coefficients for
each language user separately by treating those language users as
individual predictors. Because we also included all the predictors from
Sevenants (2024a) in our analysis, these more general predictors
(gender, dialect region, distance from northern Antwerp, formality) form
the ``expectation'' for each specific speaker. Due to the multifactorial
control characteristic\sidenote{\footnotesize Multifactorial control means that the
  analysis takes all predictors into account at the same time. This
  guarantees that variation due to one predictor is not assumed to stem
  from another.} of regression analysis, this makes it so each
individual speaker's coefficient becomes a measure of the variation that
is not explained on the basis of the more general predictor terms,
i.e.~the individual deviation or variation we expressed our desire for
in the previous sections.

For an overview of all predictors included in our model, see
Table~\ref{tbl-predictors}.

\begin{longtable}[]{@{}
  >{\raggedright\arraybackslash}p{(\columnwidth - 2\tabcolsep) * \real{0.5000}}
  >{\raggedright\arraybackslash}p{(\columnwidth - 2\tabcolsep) * \real{0.5000}}@{}}
\toprule\noalign{}
\begin{minipage}[b]{\linewidth}\raggedright
predictor
\end{minipage} & \begin{minipage}[b]{\linewidth}\raggedright
explanation
\end{minipage} \\
\midrule\noalign{}
\endfirsthead
\toprule\noalign{}
\begin{minipage}[b]{\linewidth}\raggedright
predictor
\end{minipage} & \begin{minipage}[b]{\linewidth}\raggedright
explanation
\end{minipage} \\
\midrule\noalign{}
\endhead
\bottomrule\noalign{}
\endlastfoot
dialect* & the dialect area the tweet poster belongs to (``West
Flemish'', ``East Flemish'', ``Brabantian'', ``Limburgish'') \\
gender & the estimated gender of the tweet poster -- generated
automatically \\
distance from northern Antwerp & the distance of the poster's location
in km from the northern Antwerp area \\
formality & the formality of the tweet -- derived from whether the tweet
author interacts with another user in the tweet \\
user\_id* & a unique (anonymised) identifier for the tweet author --
allows us to know which tweets belong to the same person \\
\caption{An overview of all predictors which will be used in the elastic
net regression. Predictors marked with * receive separate predictors for
each level (for example, separate predictors for each tweet
author).}\label{tbl-predictors}\tabularnewline
\end{longtable}

For our analysis, we used ElasticToolsR (Sevenants 2023), a package for
the R language (R Core Team 2021) which simplifies running elastic net
regression models.

\section{Results}\label{results}

In this section, we will go over the results produced by our elastic net
regression model. We will first look at the predictors included in the
previous study to ensure our model produces comparable results. Then, we
will dive into the predictors describing the behaviour of the individual
language users.

\subsection{Controlling predictors}\label{sec-controlling-predictors}

Table~\ref{tbl-other-coefficients} gives an overview of the ``control''
coefficients that were included in order to have multifactorial control
for the individual variation. Since the innovative \emph{bent} variant
was encoded as outcome ``1'', and the conservative \emph{zijt} was
encoded as outcome ``0'', positive coefficients mean a a correction
towards the innovative variant, while negative corrections mean a
correction towards the conservative variant. These corrections must be
interpreted from the situation expressed in the intercept, a ``reference
situation'' that is defined as follows:

\begin{itemize}
\tightlist
\item
  formal situation (no reply or interaction)
\item
  at distance ``0'' from northern Antwerp
\item
  for users assumed to be male
\end{itemize}

In general, the control variable results are comparable to the results
from Sevenants (2024a). The innovative \emph{gij bent} is more popular
in the Brabant dialect area (positive correction), and less popular in
the other areas (negative corrections). We even see the same anomaly for
the West-Flemish dialect area.\sidenote{\footnotesize The results for the West-Flemish
  dialect area need to be interpreted in conjunction with the ``distance
  from northern Antwerp'' parameter. The regression analysis assumes a
  zero distance from Antwerp from the intercept, but this does not make
  sense for dialect areas other than Brabant. Therefore, if we apply the
  extra corrections expressed in ``distance from northern Antwerp'', we
  get the expected negative correction towards the conservative form
  (-2.079). Still, the \emph{gij} pronoun is quite rare in this area, so
  the West-Flemish results should be treated with caution.} \emph{gij
bent} becomes less popular as we move away from northern Antwerp, and
more informal tweets also correct towards the conservative \emph{gij
zijt} (albeit only very slightly). We still see the expected gender
effect, where tweets assumed to come from women have a higher
probability of the use of \emph{gij bent}. Compared to Sevenants
(2024a), some predictors have more subtle effects. Especially the
distance correction and the formality distinction correction are
smaller. We assume these differences stem from (i) the coefficient
attenuation from the elastic net penalty (ii) a difference in dataset
size -- Sevenants (2024a) used only one tweet per user because of issues
with random effects, which we sidestep by using elastic net regression
(iii) parts of the variation attributed to other predictors in Sevenants
(2024a) will now be attributed to individual variation. Still, broadly
speaking, the results remain comparable. Therefore, we can safely move
on to the discussion of the individual variation.

\begin{longtable}[]{@{}lrl@{}}

\toprule\noalign{}
predictor & coefficient & retained \\
\midrule\noalign{}
\endhead
\bottomrule\noalign{}
\endlastfoot
dialect: Brabants & 0.878 & yes \\
gender: female & 0.705 & yes \\
dialect: West-Vlaams & 0.000 & no \\
distance from Antwerp (km) & -0.017 & yes \\
dialect: Oost-Vlaams & -0.081 & yes \\
reply: yes & -0.098 & yes \\
dialect: Limburgs & -1.216 & yes \\


\caption{\label{tbl-other-coefficients}Overview of the controlling
predictors predictors, their values and whether they were retained by
the elastic net regression}

\tabularnewline
\end{longtable}

\subsection{Individual variation}\label{individual-variation}

In this section, we will go over the coefficients that were computed for
each language user. We will first look at the variation patterns
globally, and then zoom in on the differences between different groups
of language users. We will also attempt to model the coefficients in
order to find relations between them.

\subsubsection{Global variation pattern}\label{global-variation-pattern}

First, we will look at the coefficient globally. The density plot of the
variation coefficients in Figure~\ref{fig-general-density} gives a
global overview of the how each language user ``deviates'' from what is
expected given their gender, location etc.. Negative coefficients on the
left side of the plot show deviations of individual language users
towards the conservative \emph{gij zijt}. Positive coefficients on the
right side of the plot show deviations of individual language users
towards the innovative \emph{gij bent}. The middle dotted line marks the
zero coefficient, i.e.~those who do not deviate from the expected norm.
We see that there is a bi-modal distribution in the variation
coefficients. The first distribution shows many slights corrections
towards the conservative form (first bump). The second distribution
shows a smaller number of larger corrections towards the innovative form
(second bump). This bi-modal variation pattern would have remained
invisible in a more traditional regression model.

Note that each of the two different distributions might not consist of a
homogeneous type of people. For example, the innovative corrections
might include people in regions where the innovative form is rarely
used, and who use the innovative form nonetheless. At the same time,
innovative corrections might also stem from people in an already
innovative area who ``out-innovate'' the expectations, using the
innovative form even more often than is expected. We will attempt to
tease out this distinction in the next section.

\begin{figure}

\centering{

\includegraphics{paper_files/figure-pdf/fig-general-density-1.pdf}

}

\caption{\label{fig-general-density}A density plot describing the
individual variation coefficients in our dataset. The left side of the
plot (negative coefficients) shows corrections towards the conservative
form, the right side of the plot (positive coefficients) shows
corrections towards the innovative form.}

\end{figure}%

\subsubsection{Variation patterns among
groups}\label{sec-group-variation}

\paragraph{Gender}\label{gender}

Next, we will look at the patterns of variation among different groups.
Figure~\ref{fig-gender-violin} shows the distribution of variation
coefficients between men and women. There are two key points. First, we
see that men and women have comparable profiles, with the same bi-modal
trend seen in the general coefficients overview. We will address this
bi-modality in the section on dialect variation patterns.

Secondly, we see that the most extreme innovative corrections in the
plot stem from men, while the most extreme conservative corrections stem
from women. This is peculiar, in the sense that this pattern
``reverses'' what we know from from Sevenants (2024a) and
Section~\ref{sec-controlling-predictors}. However, keep in mind that the
plot shows \emph{corrections} on top of \emph{expectations}. Though
there are general trends that dictate a certain preference for men or
women, this does not mean that \emph{all} individuals from that group
follow that same pattern. Figure~\ref{fig-gender-violin} shows this
reversal: there are ample outlier speakers who deviate from what is
typically expected from their gender, and this also highlights the need
to be cautious when making generalisations in this regard. At the same
time, the distributions between men and women are still similar enough
that we likely cannot speak of distinctly different profiles. We will
come back to this in our regression model of the coefficients in
Section~\ref{sec-regression}.

\begin{figure}

\centering{

\includegraphics{paper_files/figure-pdf/fig-gender-violin-1.pdf}

}

\caption{\label{fig-gender-violin}A violin plot describing the
individual variation coefficients for men and women. A wider figure
means more data for that y value.}

\end{figure}%

\paragraph{Dialect}\label{dialect}

Next, we look at the dispersion of variation by Flemish dialect region
in Figure~\ref{fig-region-violin}. Here, we do not see a ``reversal''
pattern, in contrast to Figure~\ref{fig-gender-violin}. There seem to be
two variation profiles: a Brabant variation profile, and a non-Brabant
variation profile. The Brabant variation profile shows how in the
Brabant dialect region, there is a slight preference for innovation at
the upper end of the coefficient values, with declining coefficents
towards conserative corrections. In the three non-Brabant regions, we
see that there are some innovators, but they are generally few. In these
dialects, the values gravitate towards more conservative corrections.
With these results, we have an explanation for the bi-modal nature of
Figure~\ref{fig-general-density}: the first distribution, centred
\emph{below} zero, consists mostly of language users from non-Brabant
dialect areas who are even more conservative than we expect from their
region, and therefore correct ``downwards''. The second distribution,
centred \emph{above} zero, consists mostly of eager innovators from the
already innovative Brabant dialect area. As avid users of the innovative
form, they correct ``upwards'' in our plots.

\begin{figure}

\centering{

\includegraphics{paper_files/figure-pdf/fig-region-violin-1.pdf}

}

\caption{\label{fig-region-violin}A violin plot describing the
individual variation coefficients by region. A wider figure means more
data for that y value.}

\end{figure}%

We see in Figure~\ref{fig-region-violin} that, while they are few,
innovators in non-Brabant dialect areas \emph{do} exist. One might
wonder what the profile of these rare innovators is. Perhaps they they
live close to the border in their specific dialect regions, and are
therefore ``atypically'' innovative for their region. To answer this
question, we built a Generalised Additive Model or GAM (Hastie and
Tibshirani 1987). A GAM can fit smooth curves to ``wavy'' data. If
extrapolated to two dimensions, it allows one to create
``weatherman-like'' heatmaps for linguistic data. In this case, we
modelled the elastic net coefficients on the basis of the latitude and
longitude values associated with the places in the profiles of the
Twitter users, in addition to the interaction of the two axes (``tensor
smooth'', \(te\)). This gives rise to the following equation:

\[
\text{construction} \sim s(\text{longitude}) + s(\text{latitude}) + te(\text{longitude}, \text{latitude})
\]

We will immediately shift towards the visual interpretation of the GAM
in TODO. Dark blue colours show corrections towards the conservative
variant, whereas light yellow colours show corrections towards the
innovative variant. Beware to keep in mind, however, that these
corrections are to be interpreted against the expected behaviour in that
specific dialect region. In the GAM plot, we see that the northern
Antwerp region is specifically innovative in contrast to an already
innovative dialect region, Brabant. This is expected from the results by
Sevenants (2024a). What is especially interesting given our interest in
the innovators in the non-Brabant regions, is that we indeed see that
areas closer to the Brabant border show corrections that tend more
towards the innovative variant. In addition, we also see that there
seems to be an additional horizontal diffusion pattern, where the more
innovative variant seems to spread from east to west in the east and
from west to east in the west. This pattern remained invisible in the
original study. Computer simulations have shown that border-adjacent
speakers can play an important role as gatekeepers of innovations, since
their hybrid identity can form a ``bridge'' through which innovations
can be passed on (Sevenants and Speelman 2021). In a sense, the
horizontal positive variation coefficient pattern shows the pathway that
\emph{gij bent} might take in the future in order to break through in
these other dialect regions.

TODO figure

In general, then, we see that there are two trends in the variation
coefficients. Firstly, for gender, we see a ``reversal'' pattern, where
typical behaviour is not shared across the entire gender group.
Secondly, we see a wave-like (François 2014) horizontal diffusion
pattern in the non-Brabant dialect regions, even in regions which, in
the general analysis, are thought to be rather conservative. Though
speakers are of course free to use whatever forms they want, how they
deviate from expectations is not random.

{[}violin plots{]}

\subsubsection{Network effects}\label{sec-regression}

As mentioned in Section~\ref{sec-introduction}, the networked aspect of
social media data provides an opportunity for researchers to investigate
which people use what variant, and what that says about the diffusion of
a new variant. Since our dataset comes from the Twitter platform, we
also dispose of metrics which define a user's relative popularity on the
platform. The first metric, ``Followers'', tallies how many other users
want to see a user's posts appear in their own social media feed. Users
with a large number of followers have a broad reach, and therefore have
the possibility to be influential in their language use. The second
metric, ``Following'', states how many other people's posts a user wants
to see in their social media feed. Users who follow a large number of
other users have the opportunity of being exposed to a larger linguistic
pallette -- though this need not necessarily be the case.

To assess the influence of the users in our dataset within the larger
Twitter network, we devised a metric which is based on the ratio between
``Followers'' and ``Following''. The idea of the metric is that users
who have a large number of followers, yet follow few people themselves,
have a larger relative influence than users without this asymmetry. Our
influence metric is shown in the following equation:

\[
\log{
  \frac{
    \text{\# followers} + 0.001
  } {
    \text{\# following} + 0.001
  }
}
\]

Our influence metric is based on the priming ratio from Sevenants, Van
de Velde, and Speelman (n.d.) and hinges on logarithm and Laplace
smoothing (Brysbaert and Diependaele 2013). The logarithm is a key
operation in order to compress the wide numeric range found in follower
and following counts. Where some users have thousands of followers, some
only have a few. The use of the logarithm scales this discrepancy down.
In addition, we use Laplace smoothing in order to avoid division by
zero, in the event that users do not follow other users. The mean
influence value among our speakers is 0.30 (median 0.28). The boxplot in
Figure~\ref{fig-influence} shows that our dataset generally consists of
more influential figures, i.e.~users who have more followers than they
themselves follow people. This is expected, since generally speaking,
90\% of social media users only read, but never post themselves (Nielsen
2006). Since our dataset consists of posts, it makes sense that our
dataset mainly engages with the other 10\%, the active users.

\begin{figure}

\centering{

\includegraphics{paper_files/figure-pdf/fig-influence-1.pdf}

}

\caption{\label{fig-influence}}

\end{figure}%

In our case, we want to know how the language users who are thought to
be influential in our network behave. Are they are the forefront of
innovation, or are they conservative? Depending on what profile we find,
we might be able to discern in what phase of diffusion the \emph{gij
bent} innovation is situated.

To answer this question, we built a common linear regression model. We
model the coefficients in term of the influence of the users, but we
also supplement our analysis with the dialect and gender distinctions we
discussed earlier. Bundling all these possible influences in the same
model will ensure that we do not accidentally assign variation between
language users to the wrong source. It will also function as an extra
check for the results from Section~\ref{sec-group-variation}, as we will
now be able to check statistically whether the distinctions we spotted
in the visual exploration are statistically significant distinctions.
Table~\ref{tbl-linear-predictors} gives an overview of the predictors
that were used in the linear regression model.

\begin{longtable}[]{@{}
  >{\raggedright\arraybackslash}p{(\columnwidth - 2\tabcolsep) * \real{0.5000}}
  >{\raggedright\arraybackslash}p{(\columnwidth - 2\tabcolsep) * \real{0.5000}}@{}}
\toprule\noalign{}
\begin{minipage}[b]{\linewidth}\raggedright
predictor
\end{minipage} & \begin{minipage}[b]{\linewidth}\raggedright
explanation
\end{minipage} \\
\midrule\noalign{}
\endfirsthead
\toprule\noalign{}
\begin{minipage}[b]{\linewidth}\raggedright
predictor
\end{minipage} & \begin{minipage}[b]{\linewidth}\raggedright
explanation
\end{minipage} \\
\midrule\noalign{}
\endhead
\bottomrule\noalign{}
\endlastfoot
dialect & the dialect area the tweet poster belongs to (``West
Flemish'', ``East Flemish'', ``Brabantian'', ``Limburgish'') \\
gender & the estimated gender of the tweet poster -- generated
automatically \\
influence & a metric expressing the relative influence of a language
user within the Twitter network \\
\caption{An overview of all predictors used in the linear regression
model.}\label{tbl-linear-predictors}\tabularnewline
\end{longtable}

\begin{longtable}[]{@{}llllll@{}}

\toprule\noalign{}
term & coefficient & std. error & t value & p value & \\
\midrule\noalign{}
\endhead
\bottomrule\noalign{}
\endlastfoot
(Intercept) & 1.15 & 0.19 & 6.03 & \textless0.01 & * \\
gender: female & -0.31 & 0.21 & -1.46 & 0.1455 & \\
dialect: Limburgs & -1.37 & 0.35 & -3.87 & \textless0.01 & * \\
dialect: Oost-Vlaams & -2.04 & 0.28 & -7.23 & \textless0.01 & * \\
dialect: West-Vlaams & -1.41 & 0.33 & -4.29 & \textless0.01 & * \\
influence & 0.31 & 0.12 & 2.58 & 0.0103 & * \\


\caption{\label{tbl-linear-predictors}Results of the linear regression
analysis. * indicates statistical significance}

\tabularnewline
\end{longtable}

We will first go over the controlling variables we discussed earlier,
and then move on to the influence predictor.

\begin{enumerate}
\def\labelenumi{\arabic{enumi}.}
\tightlist
\item
  Intercept
\end{enumerate}

First, we look at the intercept of the model. Just like in
Section~\ref{sec-controlling-predictors}, the intercept here expressed
the ``default'' situation against which further corrections need to be
understood. Here, the intercept expresses a language user \ldots{}

\begin{itemize}
\tightlist
\item
  \ldots{} from the Brabant dialect area
\item
  \ldots{} who is assumed to be male
\item
  \ldots{} with an influence of ``zero''
\end{itemize}

Our intercept starts from a value of 1.154, which shows a slight bias
for innovative behaviour. All further corrections need to be understood
relative to this value.

\begin{enumerate}
\def\labelenumi{\arabic{enumi}.}
\setcounter{enumi}{1}
\tightlist
\item
  Gender
\end{enumerate}

We see that the distinction between users who are thought to be male and
users who are thought to be female does not reach the significance
threshold. Therefore, the differences which we discussed in
Section~\ref{sec-group-variation} are not different enough to the degree
that they are actually distinct. Still, the fact that the
\emph{tendency} is clearly there is still interesting, and the reversal
of the extreme values of course still holds.

\begin{enumerate}
\def\labelenumi{\arabic{enumi}.}
\setcounter{enumi}{2}
\tightlist
\item
  Dialect area
\end{enumerate}

Next, we also see that there are significant differences between the
reference dialect area (Brabant) and all other dialect areas. As was
already clear from the visual inspection of the corrections per dialect
area, we see that in non-Brabant areas, the corrections tend towards the
conservative form, whereas Brabant area corrections are mostly
innovative.

In addition, we wanted to know whether there are also differences
\emph{between} the non-Brabant dialact areas. To investigate this, we
ran a post-hoc analysis through emmeans (Lenth 2024). In this post-hoc
analysis, we check for each dialect area pair whether there are
significant differences in individual variation between the two. We see
in Table~\ref{tbl-posthoc-test} that there are \emph{no} further
differences between the non-Brabant dialect areas. Our assessment of the
two variation profiles was thus correct.

\begin{longtable}[]{@{}
  >{\raggedright\arraybackslash}p{(\columnwidth - 10\tabcolsep) * \real{0.3824}}
  >{\raggedright\arraybackslash}p{(\columnwidth - 10\tabcolsep) * \real{0.1765}}
  >{\raggedright\arraybackslash}p{(\columnwidth - 10\tabcolsep) * \real{0.1618}}
  >{\raggedright\arraybackslash}p{(\columnwidth - 10\tabcolsep) * \real{0.1176}}
  >{\raggedright\arraybackslash}p{(\columnwidth - 10\tabcolsep) * \real{0.1176}}
  >{\raggedright\arraybackslash}p{(\columnwidth - 10\tabcolsep) * \real{0.0441}}@{}}

\toprule\noalign{}
\begin{minipage}[b]{\linewidth}\raggedright
contrast
\end{minipage} & \begin{minipage}[b]{\linewidth}\raggedright
coefficient
\end{minipage} & \begin{minipage}[b]{\linewidth}\raggedright
std. error
\end{minipage} & \begin{minipage}[b]{\linewidth}\raggedright
t value
\end{minipage} & \begin{minipage}[b]{\linewidth}\raggedright
p value
\end{minipage} & \begin{minipage}[b]{\linewidth}\raggedright
\end{minipage} \\
\midrule\noalign{}
\endhead
\bottomrule\noalign{}
\endlastfoot
Brabants ↔ Limburgs & 1.37 & 0.35 & 3.87 & \textless0.01 & * \\
Brabants ↔ Oost-Vlaams & 2.04 & 0.28 & 7.23 & \textless0.01 & * \\
Brabants ↔ West-Vlaams & 1.41 & 0.33 & 4.29 & \textless0.01 & * \\
Limburgs ↔ Oost-Vlaams & 0.67 & 0.42 & 1.61 & 0.3713 & \\
Limburgs ↔ West-Vlaams & 0.04 & 0.45 & 0.09 & 0.9998 & \\
Oost-Vlaams ↔ West-Vlaams & -0.64 & 0.40 & -1.60 & 0.3788 & \\


\caption{\label{tbl-posthoc-test}Results of the \texttt{emmeans} posthoc
analysis for dialect area. * indicates statistical significance}

\tabularnewline
\end{longtable}

\begin{enumerate}
\def\labelenumi{\arabic{enumi}.}
\setcounter{enumi}{3}
\tightlist
\item
  Network influence
\end{enumerate}

Finally, we look at the network influence parameter. Recall that our
influence predictor expresses the (im)balance between the ``followers''
and ``following'' metrics, with those users who are followed by more
people than they follow themselves being attributed a larger influence.
The regression table in Table~\ref{tbl-linear-predictors} shows that, as
our influence metric increases, so does the probability for the use of
\emph{gij bent}. This means that the more influential a speaker is in
the Twitter network, the more likely they are to use the new, innovative
\emph{gij bent} form. {[}Todo Julie uitleg hoe en waarom dit zo is{]}

{[}network effect explanation{]}

\section{Conclusion}\label{conclusion}

\begin{itemize}
\tightlist
\item
  {[}algemeen idee van de individuele variatie{]}
\item
  {[}conclusies voor tussentaal{]}
\item
  {[}idee dat je op deze manier ook kunt vaststellen in welke fase van
  de verspreiding een fenomeen zit{]}
\end{itemize}

\phantomsection\label{refs}
\begin{CSLReferences}{1}{0}
\bibitem[\citeproctext]{ref-auer_europe_2005}
Auer, Peter. 2005. {``Europe's Sociolinguistic Unity, or: A Typology of
European Dialect/Standard Constellations.''} \emph{Perspectives on
Variation: Sociolinguistic, Historical, Comparative} 7: 7--42.
\url{https://doi.org/10.1515/9783110909579}.

\bibitem[\citeproctext]{ref-barabasi_emergence_1999}
Barabasi, Albert-Laszlo. 1999. {``Emergence of Scaling in Random
Networks.''} \emph{arXiv.org} 286 (5439): 509--12.
\url{https://doi.org/10.1126/science.286.5439.509}.

\bibitem[\citeproctext]{ref-brysbaert_dealing_2013}
Brysbaert, Marc, and Kevin Diependaele. 2013. {``Dealing with Zero Word
Frequencies: {A} Review of the Existing Rules of Thumb and a Suggestion
for an Evidence-Based Choice.''} \emph{Behavior Research Methods} 45
(2): 422--30.

\bibitem[\citeproctext]{ref-ellis_what_2012}
Ellis, Nick C. 2012. {``What Can We Count in Language, and What Counts
in Language Acquisition, Cognition, and Use?''} In \emph{Frequency
{Effects} in {Language} {Learning} and {Processing}}, edited by Stefan
Th. Gries and Dagmar Divjak, 1:7--34. De Gruyter Mouton.
\url{https://doi.org/10.1515/9783110274059.7}.

\bibitem[\citeproctext]{ref-francois_trees_2014}
François, Alexandre. 2014. {``Trees, {Waves} and {Linkages}: {Models} of
{Language} {Diversification}.''} In \emph{The {Routledge} {Handbook} of
{Historical} {Linguistics}}, edited by Claire Bowern and Evans Bethwyn,
161--89. Routledge. \url{https://doi.org/10.4324/9781315794013.ch6}.

\bibitem[\citeproctext]{ref-friedman_regularization_2010}
Friedman, Jerome, Robert Tibshirani, and Trevor Hastie. 2010.
{``Regularization {Paths} for {Generalized} {Linear} {Models} via
{Coordinate} {Descent}.''} \emph{Journal of Statistical Software} 33
(1): 1--22. \url{https://doi.org/10.18637/jss.v033.i01}.

\bibitem[\citeproctext]{ref-geeraerts_een_2000}
Geeraerts, Dirk. 2000. {``Een Zondagspak? {Het} {Nederlands} in
{Vlaanderen}: Gedrag, Beleid, Attitudes.''} \emph{Ons Erfdeel} 44:
337--44.
\url{https://wwwling.arts.kuleuven.be/qlvl/PDFPublications/01Eenzondagspak.pdf}.

\bibitem[\citeproctext]{ref-geeraerts_lectal_2005}
---------. 2005. {``Lectal Variation and Empirical Data in Cognitive
Linguistics.''} In \emph{Cognitive Linguistics: Internal Dynamics and
Interdisciplinary Interaction}, 32:163--89. Cognitive Linguistics
Research. Berlin: Mouton de Gruyter.
\url{https://doi.org/10.1515/9783110197716.2.163}.

\bibitem[\citeproctext]{ref-ghyselen_verticale_2016}
Ghyselen, Anne-Sophie. 2016. {``{Verticale structuur en dynamiek van het
gesproken Nederlands in Vlaanderen: een empirische studie in Ieper, Gent
en Antwerpen}.''} PhD thesis, {Ghent University}.

\bibitem[\citeproctext]{ref-grafmiller_general_2018}
Grafmiller, Jason, Benedikt Szmrecsanyi, Melanie Röthlisberger, and
Benedikt Heller. 2018. {``General Introduction: {A} Comparative
Perspective on Probabilistic Variation in Grammar.''} \emph{Glossa: A
Journal of General Linguistics} 3 (1).
\url{https://doi.org/10.5334/gjgl.690}.

\bibitem[\citeproctext]{ref-gries_most_2015}
Gries, Stefan Thomas. 2015. {``The Most Under-Used Statistical Method in
Corpus Linguistics: Multi-Level (and Mixed-Effects) Models.''}
\emph{Corpora} 10 (1): 95--125.
\url{https://doi.org/10.3366/cor.2015.0068}.

\bibitem[\citeproctext]{ref-hastie_generalized_1987}
Hastie, Trevor, and Robert Tibshirani. 1987. {``Generalized {Additive}
{Models}: {Some} {Applications}.''} \emph{Journal of the American
Statistical Association} 82 (398): 371--86.
\url{https://doi.org/10.1080/01621459.1987.10478440}.

\bibitem[\citeproctext]{ref-impe_vlamingen_2007}
Impe, Leen, and Dirk Speelman. 2007. {``Vlamingen En Hun (Tussen) Taal:
{Een} Attitudineel Mixed Guise-Onderzoek.''}
\emph{Handelingen-Koninklijke Zuid-Nederlandse Maatschappij Voor Taal-En
Letterkunde En Geschiedenis} 61.

\bibitem[\citeproctext]{ref-keller_language_1994}
Keller, Rudi. 1994. \emph{On {Language} {Change}: {The} {Invisible}
{Hand} in {Language}}. Psychology Press.

\bibitem[\citeproctext]{ref-kretzschmar_language_2015}
Kretzschmar, William A. 2015. \emph{Language and {Complex} {Systems}}.
Cambridge: Cambridge University Press.
\url{https://doi.org/10.1017/CBO9781316179017}.

\bibitem[\citeproctext]{ref-labov_study_1968}
Labov, William et al. 1968. {``A Study of the Non-Standard English of
Negro and Puerto Rican Speakers in New York City.''}

\bibitem[\citeproctext]{ref-emmeans}
Lenth, Russell V. 2024. {``Emmeans: Estimated Marginal Means, Aka
Least-Squares Means.''} \url{https://github.com/rvlenth/emmeans}.

\bibitem[\citeproctext]{ref-milroy_belfast_1978}
Milroy, James, and Lesley Milroy. 1978. {``Belfast: Change and Variation
in an Urban Vernacular.''} \emph{Sociolinguistic Patterns in British
English} 19: 19--36.

\bibitem[\citeproctext]{ref-nielsen_participation_2006}
Nielsen, Jakob. 2006. {``Participation {Inequality}: {The} 90-9-1 {Rule}
for {Social} {Features}.''} \emph{Nielsen Norman Group}.
\url{https://www.nngroup.com/articles/participation-inequality/}.

\bibitem[\citeproctext]{ref-r_language_2021}
R Core Team. 2021. {``R: A Language and Environment for Statistical
Computing.''} Vienna, Austria: R Foundation for Statistical Computing.
\url{https://www.R-project.org/}.

\bibitem[\citeproctext]{ref-sevenants_elastic_2024}
Sevenants, Anthe. 2023. {``ElasticToolsR.''} Zenodo.
\url{https://doi.org/10.5281/zenodo.8113291}.

\bibitem[\citeproctext]{ref-sevenants_zijt_2024}
---------. 2024a. {``{`{Zijt} Gij Dat of Bent Gij Dat?'}: {Een}
Alternantiestudie van de Tweede Persoon Enkelvoud van Zijn in {Vlaamse}
Tussentaal,''} January.
\url{https://doi.org/10.5117/TET2023.2.003.SEVE}.

\bibitem[\citeproctext]{ref-sevenants_gender_2024}
---------. 2024b. {``Gender-from-Name-r.''} Zenodo.
\url{https://doi.org/10.5281/zenodo.14013816}.

\bibitem[\citeproctext]{ref-sevenants_neighbours_2021}
Sevenants, Anthe, and Dirk Speelman. 2021. {``Keeping up with the
Neighbours-an Agent-Based Simulation of the Divergence of the Standard
Dutch Pronunciations in the Netherlands and Belgium.''}
\emph{Computational Linguistics in the Netherlands Journal} 11: 5--26.
\url{https://clinjournal.org/clinj/article/view/118}.

\bibitem[\citeproctext]{ref-sevenants_elastic_to_appear}
Sevenants, Anthe, Freek Van de Velde, and Dirk Speelman. n.d.
{``Investigating Lexical-Semantic Effects on Morphosyntactic Variation
Using Elastic Net Regression.''} \emph{Corpus Linguistics and Linguistic
Theory}, Accepted.

\bibitem[\citeproctext]{ref-taeldeman_zich_2008}
Taeldeman, Johan. 2008. {``Zich Stabiliserende Grammaticale Kenmerken in
{Vlaamse} Tussentaal.''} \emph{Taal En Tongval} 60 (1): 26--50.

\bibitem[\citeproctext]{ref-vandekerckhove_cbd_2005}
Vandekerckhove, Reinhild. 2005. {``Belgian Dutch Versus Netherlandic
Dutch: New Patterns of Divergence? On Pronouns of Address and
Diminutives.''} \emph{Multilingua} 24 (4): 379--97.
\url{https://doi.org/10.1515/mult.2005.24.4.379}.

\bibitem[\citeproctext]{ref-vandekerckhove_perception_2007}
Vandekerckhove, Reinhild, and Poi Cuvelier. 2007. {``The Perception of
Exclusion and Proximity Through the Use of {Standard} {Dutch},
{`Tussentaal'} and Dialect in {Flanders}.''} In \emph{Multilingualism
and Exclusion. {Policy} Practice and Prospects}, 241--56. Studies in
{Language} {Policy} in {South}-{Africa}. Pretoria: Van Schaik
Publishers.

\bibitem[\citeproctext]{ref-zipf_psycho-biology_1965}
Zipf, George Kingsley. 1965. \emph{The {Psycho}-{Biology} of
{Language}}. Cambridge, Massachusetts: MIT Press.

\end{CSLReferences}




\end{document}
